\chapter{Summary of Saleena and Paleri}
\label{chap:chapter4}

Saleena and Paleri gave an algorithm for 
\textit{global value numbering (GVN)}. GVN works by assigning a value 
number to variables and expressions. The same value number is 
assigned to those variables and expressions which are provably 
equivalent. A notable difference between Herbrand equivalence and GVN 
is that in Herbrand equivalence we talk about equivalences at a 
particular program point but in GVN are concerned with equivalence 
between expressions at two different program points.

The data structure used in the algorithm is called \textit{value 
expression}. An expression with value numbers as operands, is called 
value expression. Two expressions are equivalent if they have same 
value expression. So, a value expression can be used to represent a 
set of equivalent program expressions.

\section{Notation}
Input is a flow graph atmost one assignment statement in each node
which has one of the following forms
$$x\; ::=\; e$$ 
$$e\; ::=\; x\: |\: c\: |\: x\,op\,x$$
The flow graph also has two additional empty $ENTRY$ and $EXIT$ nodes.
For a node $n$, $IN_n$ and $OUT_n$ denotes the input and output 
program points of the node.

Expression pool at a program point, is a partition of expressions at 
that point, in which equivalent expression belongs to the same 
partition. Each class will have a value number which we will consider 
as its first element. For a node $n$, $EIN_n$ and $EOUT_n$ denotes 
the expression pools at input and output program points of the node.

\section{Value Expression}
The value expression corresponding to an expression is obtained by 
replacing actual operands with their corresponding value numbers.
Example - For the expression-pool $\{\{v_1, a, x\}, \{v_2, b, y \}\}$ 
and statement $z ::= x + y$ , the value-expression for $x + y$ will be
$v_1 + v_2$. Instead of $x + y$, its value-expression is included in 
the expression-pool, with a new value number ie. the new 
expression-pool would be 
$\{\{v_1, a, x\},\{v_2, b, y\}, \{v_3, v_1 + v_2, z\}\}$.

The value expression $v_1 + v_2$ represents not just $x + y$ but the
set of equivalent expressions $\{a + b, x + b, a + y, x + y\}$. Its 
presence indicates that an expression from this set is already 
computed and this information is enough for detection of redundant 
computations. Also, a single binary value expression can represent 
equivalence among any numbre of expressions of any length. Example - 
$v_1 + v_3$ represents, $a + z$, $x + z$, $a + (a + b)$, 
$a + (x + b)$ and so on.

\section{Algorithm}
Similar to Gulwani's algorithm, the algorithm consists of two main 
functions - a transfer function for changes in expression pool across 
assignment statements and a confluence function to find the 
expression pool at points were two branches meet. The algorithm 
starts with $EOUT_{ENTRY} = \phi$, and uses transfer and confluence 
functions to calculate expression pools at other points. This process 
is repeated till there is any change in the equivalence information.
For detailed implementation refer to \cite{Saleena}.
