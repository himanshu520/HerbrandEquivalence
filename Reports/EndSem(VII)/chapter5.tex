\chapter{Summary of Babu, Krishnan and Paleri}
\label{chap:chapter5}

One of the problems with other former approaches to Herbrand 
equivalence is that most of the alogrithms were based on fix point 
computations. But the classical definition of Herbrand equivalence is 
not a fix point based definition making it difficult to prove their 
precision or completeness. Babu, Krishnan and Paleri \cite{Babu} gave 
a new lattice theoretic formulation of Herbrand equivalences and 
proved its equivalence to the classical version.

The paper defines a congruence relation on the set of all possible 
expressions and shows that the set of all congruences for a complete 
lattice. Then for a given dataflow framework with $n$ program points, 
a continuous composite transfer function is defined over the n-fold 
product of the above lattice such that the maximum fix point of the 
function yields the set of Herbrand equivalence classes at various 
program points. Finally, equivalence of this approach to the 
classical meet over all path definition of Herbrand Equivalence is 
established.

Below is a brief summary of the developments in the paper, for more 
detailed approach and proofs and for equivalence to MOP  
characterization refer to \cite{Babu}.

\section{Program Expressions}

Let $C$ and $X$ be the set of constants and variables occurring in 
the program respectively. The program expressions (terms) can be 
described as 
$$t\; ::=\; c\; |\; x\; |\; t_1 + t_2$$
where $c \in C$ and $x \in X$.

\section{Congruence Relation}
Let $T$ be the set of all program terms. A partition $P$ of terms in $T$ is said to be a congruence (of terms) if 
\begin{itemize}
    \item For $t$, $t'$, $s$, $s'$ $\in$ $T$, $t' \cong t$ and $s' \cong s$ iff $t' + s' \cong t + s$. 
    \item For $c \in C$, $t \in T$, if $t \cong c$ then either $t = c$ or $t \in X$.
\end{itemize}
Let $G(T)$ be the set of all congruences over $T$. We say $P_1 
\preceq P_2$ for $P_1, P_2 \in G(T)$, if 
$\forall A_1 \in P_1, \exists A_2 \in P_2$ such that 
$A_1 \subseteq A_2$. We define \textit{confluence} operation as 
$$P_1 \land P_2\; =\; \{A_i \cap B_j\: |\: A_i \in P_1 \text{ and } B_j \in P_2\}$$
Now, we extend $G(T)$ to $\overline{G(T)}$ by introducing abstract 
congruence $\top$ satisfying 
$P \land \top = \top, \forall P \in \overline{G(T)}$.
Also, we denote the congruence in which every element is in a 
separate class as $\bot$. 

$(\overline{G(T)}, \preceq, \bot, \top)$ 
forms a complete lattice, with $\land$ as meet operator.

\section{Transfer function}
An assignment $y := \beta$ transforms a congruence $P$ to another 
congruence $P'$. This can be described in the form of transfer 
function $f_{y = \beta}:G(T) \to G(T)$, given by
\begin{itemize}
    \item $B_i = \{t \in T\ |\ t[y \leftarrow \beta] \in A_i\}, \text{ for each } A_i \in P$
    \item $f_{y = \beta}(P) = \{B_i\ | \ B_i \neq \phi\}$
\end{itemize}
We extend this definition to form extended transfer function, 
$\overline{f}_{y=\beta} : \overline{G(T)} \to \overline{G(T)}$ 
by defining $\overline{f}_{y=\beta}(\top)\ =\ \top$, otherwise $\overline{f}_{y=\beta}(P) = f_{y=\beta}(P)$.
The extended transfer function is distributive, monotonic and continuous.

\section{Non deterministic assignment}
An assignment $y := *$ transforms a congruence $P$ to another 
congruence $P'$. This can be described in the form of another 
transfer function $f_{y = *}:G(T) \to G(T)$, given by: 
for every $t$, $t' \in T$, $t \cong_{f(P)} t'$, (here $f(P) = f_{y = *}(P)$ for simplicity) iff
\begin{itemize}
    \item $t \cong_P t'$
    \item $\forall \beta \in (T \setminus T(y)),\ t[y \leftarrow \beta] \cong_p t'[y \leftarrow \beta]$
\end{itemize}
As before we extend this transfer function to 
$\overline{f}_{y=*} : \overline{G(T)} \to \overline{G(T)}$ by defining
$\overline{f}_{y=*}(\top)\ =\ \top$, otherwise 
$\overline{f}_{y=*}(P) = f_{y=*}(P)$. 
The function $\overline{f}_{y=*}$ is also continuous.

\section{Dataflow analysis Framework}
A dataflow framework over $T$ is $D = (G, F)$ where $G(V, E)$ is the 
control flow graph associated with the program and $F$ is a 
collection of transfer function associated with program points.

\section{Herbrand Equivalence}
The Herbrand Congruence function $H_D : V(G) \to \overline{G(T)}$ 
gives the Herbrand Congruence associated with each program point and 
is defined to be the maximum fix point of the \textit{continuous
composite transfer function} 
$f_D : \overline{G(T)}^n \to \overline{G(T)}^n$, where 
$\overline{G(T)}^n$ is the product lattice, $f_D$ is a function satisfying $\pi_k \circ f_D = f_k$. Here $\pi_k$ is the projection map
and $f_k : \overline{G(T)}^n \to \overline{G(T)}$ is defined as follows 
\begin{itemize}
    \item   If k = 1, the entry point of the program $f_k = \bot$.
    \item   If k is a function point with $Pred(k) = \{j\}, \text{ then } f_k = h_k \circ \pi_j$ where 
    $h_k$ is the extended transfer function corresponding to function point k.
    \item   If k is a confluence point with $Pred(k) = {i, j}, \text{ then } f_k = \pi_{i, j}, \text{ where }
    \pi_{i, j}:\overline{G(T)}^n \to \overline{G(T)}$ is given by $\pi_{i,j}(P_1,\ \dots,\ P_n) = P_i \land P_j$.
\end{itemize}.