\chapter{Review of Prior Works}
\label{chap:chapter2}

Existing algorithms for calculating Herbrand Equivalence are either
exponential or are imprecise. The precise algorithms are based on an 
early algorithm by Kildall \cite{Kildall}, which discovers 
equivalences by performing an abstract interpretation over the 
lattice of Herbrand equivalences. Kildall algorithms is precise in 
the sense it finds all the Herbrand equivalences but is exponential 
in time. The partition refinement algorithm of Alpern, Wegman and 
Zadek (AWZ) \cite{AWZ} is efficient but is much imprecise compared to 
Kildall's. AWZ algorithm represent the values of variables after a 
join using a fresh selection function $\phi_i$, similar to functions 
in the static single assignment form and treats $\phi_i$ as 
uninterpreted functions. It is incomplete in the sense it treats all 
$\phi_i$ as uninterpreted. In an attempt to remedy this problem, 
Ruthing, Knoop and Steffen proposed a polynomial-time algorithm (RKS) 
\cite{RKS} that alternately applies the AWZ algorithm and some 
rewrite rules for normalization of terms involving $\phi$ functions, 
until the congruence classes reach a fixed point. Their algorithm 
discovers more equivalences than the AWZ algorithm, but remains 
incomplete. 

Gulwani and Necula \cite{Gulwani} gave algorithm to find the Herbrand 
Equivalence classes restricted to program expressions. There 
algorithm is linear in parameter $s$, where $s$ is the maximum times 
$+$ occurs in a program expression. Clearly $s$ can take a maximum 
value of $n$, which is the program size, so the algorithm in all is 
polynomial in $n$. Later, Saleena and Paleri \cite{Saleena} showed  
that Gulwani's algorithm losses some information as it removes a 
equivalence class if it does not contain a variable or a constant. 
The global value numbering (GVN) algorithm proposed by them was able 
to detect more redundencies compared to that by Gulwani and Necula.

One problem is that most of these alogrithms were based on fix point 
computations but the classical definition of Herbrand equivalence is 
not a fix point based definition making it difficult to prove their 
precision or completeness. Babu, Krishnan and Paleri \cite{Babu} 
developed a lattice theoretic fix-point formulation of Herbrand 
Equivalence on the lattice defined over the set of all terms 
constructible from variables, constants and operators of a program. 
They showed this definition is equivalent to the classical meet over 
all path characterization over the set of all possible expressions. 
The algorithm proposed by them is able to detect all the equivalences 
as by that of Saleena and Paleri.

So, to sum up Kildall's algorithm finds all the equivalent classes 
but is exponential. The algorithms by Saleena and Paleri; Babu, 
Krishnan and Paleri are polynomial and efficient among other 
imprecise algorithms. They are able to find all equivalence classes 
restricted to program expressions (all expressions with length atmost 
2), which is precisely what is practically useful.