\chapter{Performing Optimizations}
\label{chap:chapter8}

This chapter explains how to use the Herbrand Equivalence analysis
information to perform actual program optimizations.

\section{Available Variables}
\label{AvailableVariables}
A variable is said to be \textbf{available} at a basic block if 
it is defined at some point along all the execution paths from the 
start of the program that reaches the beginnning of that basic block.
It is forward data analysis problem and can be determined via fixed 
point computation.

The universe $\mathcal U$ would be the set of all program variables.
Also, let $\mathcal B$ be the set of all basic blocks in the program.
Denote by $DEF[B]$ the set of variables defined in the basic block
$B \in \mathcal B$, and $IN[B]$ be corresponding set of available 
variables. Initialise $OUT[B] = \mathcal U, \forall B \in \mathcal B$ 
and perform updates as follows till a fixed point is reached.
$$IN[B] = \cap_{B' \in Predecessors(B)} OUT[B']$$
$$OUT[B] = IN[B] \cup DEF[B]$$

Note that the definition of \textbf{available variables} is different
from that of \textbf{reaching definitions} and \textbf{available 
expressions}. If a variable is available at the beginning of a given
basic block, then that variable can be used in the basic block without
defining it because by \textit{definition of avilable variables} it 
would already be defined along every path from the start of the 
program to that basic block.

The definition of available variables can be extended to that of any
instruction. Variables available at an instruction $I$ is the union 
of variables available at the beginning of its basic block and any 
variables defined by the instructions preceding it in that block.

\section{Performing Optimizations}
\label{PerformingOptimizations}
The results of Herbrand Equivalence and available variables analysis 
can be used to perform optimizations involving redundant expression 
elimination and some dead code elimination.

Suppose, there is an instruction $I : z \gets e$, where $e$ is an 
expression (as already mentioned expression means either a constant, 
or a variable or a length two expression). First find the Herbrand 
equivalence class of $z$ at that program point. If the class represents
a constant valued expression ($partitions[I][z].isConst == true$), then
this instruction can be deleted and all uses of $z$ replaced by 
$partitions[I][z].constVal$, till $z$ is redefined. Else if it is not 
a constant valued expression, 
$\mathcal S = (getClass(partitions[I], z) \cap IN[I])$
will be the set of variables, such that $z$ can be replaced by 
$s \in \mathcal S$ (before $z$ and $s$ are redefined), without changing
the meaning of the program. \\ 
Pseudocode \ref{Optimize} is the precise formulation of the mentioned facts.

\begin{algorithm}
\caption{Performing Optimizations}\label{Optimize}
\begin{algorithmic}[1]
\Procedure{Optimize}{}
    \For{$B \in \mathcal B$}
        \State $Insts \gets$ \textbf{Instructions}$(B)$
        \State $AvailVars \gets IN[B]$
        \State
        \For{$I \in Insts$}
            \State $Insts.remove(I)$
            \State $z \gets$ \textbf{LValue}$(I)$
\algstore{Optimize}
\end{algorithmic}
\end{algorithm}
                    

\begin{algorithm}
\begin{algorithmic}[1]
\algrestore{Optimize}
            \If{\textbf{Defined}$(z)$}
                \State \textbf{IDstruct*} $ptr \gets partitions[I][z]$
                \State
                \If{$ptr.isConst$}
                    \State \textbf{Bool} $reachedEnd \gets true$
                    \For{$I' \in Insts$}
                        \If{\textbf{LValue}$(I') == z$}
                            \State $reachedEnd \gets false$
                            \State \textbf{break}
                        \EndIf
                        \State \textbf{Replace}$(I',\ z,\ ptr.constVal)$
                    \EndFor
                    \If{$reachedEnd$}
                        \State $B.insert(z \gets ptr.constVal)$
                    \EndIf
                    \State
                    \State \textbf{DeleteInstruction}$(I)$
                \Else
                    \State \textbf{set}$<$\textbf{Expressions}$> curPart \gets$ \textbf{GetClass}$(partitions[I],\ z)$
                    \State $curPart \gets (curPart \cap AvailVars)$
                    \State
                    \If{not $curPart.empty()$}
                        \State \textbf{Bool} $reachedEnd \gets true$
                        \State $replacement \gets curPart.first()$
                        \State
                        \For{$I' \in Insts$}
                            \If{\textbf{LValue}$(I') == z$}
                                \State $reachedEnd \gets false$           
                                \State \textbf{break}
                            \EndIf
                            \State $curPart = curPart \setminus \{$\textbf{LValue}$(I')\}$
                            \If{not $curPart.empty()$}
                                \State $replacement \gets curPart.first()$
                                \State \textbf{Replace}$(I',\ z,\ replacement)$
                            \Else
                                \State $B.insert(I', z \gets replacement)$
                                \State $AvailVars \gets AvailVars \cup \{z\}$
                                \State $reachedEnd \gets false$
                                \State \textbf{break}
                            \EndIf
                        \EndFor
                        \State
                        \If{$reachedEnd$}
                            \State $B.insert(z \gets replacement)$
                        \EndIf
                        \State \textbf{DeleteInstruction}$(I)$
                        \State
                    \Else
                        \State $AvailVars \gets AvarilVars \cup \{z\}$
                    \EndIf
                \EndIf
            \EndIf
        \EndFor
    \EndFor
\EndProcedure
\end{algorithmic}
\end{algorithm}
    
Also note the following functions whose details has not been provided.\\
\textbf{Instructions}$(B)$ :- Returns a list of instructions in the basic block $B$.\\
\textbf{Replace}$(I', z, replacement)$ :- Replaces variable $z$ in the instruction $I'$ with $replacement$ which can be a constant or another variable.\\
\textbf{B.insert}$(I)$ :- Inserts instruction $I$, at the end of basic block $B$.\\
\textbf{B.insert}$(I, I')$ :- Insert instruction $I'$ before instruction $I$ in basic block $B$.\\
\textbf{DeleteInstruction}$(I)$ :- Deletes instruction $I$ from the program.\\
\textbf{set.empty}$()$ :- Returns $true$ if $set$ is empty, else $false$.\\
\textbf{set.first}$()$ :- Returns first element from $set$.\\

\section{LLVM Implementation}
\label{LLVMImplementation}
The LLVM specific implementation of the pseudocodes mentioned can be found in 
\href{https://github.com/himanshu520/HerbrandEquivalence}{this github repository}.