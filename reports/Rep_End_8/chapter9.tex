\chapter{Proof Attempt}
\label{chap:chapter9}

\autoref{sec:AugmentedProgramApproach} contains details of the attempt made to prove the correctness of the algorithm - reason for the mentioned approach, claims made and reasons for its failure. \autoref{sec:WhyTheAlgorithmWorks} informally shows why the algorithm should work for the special case when the input program has no loops.

\section{Augmented Program Approach}
\label{sec:AugmentedProgramApproach}
Let $\mathcal{X}$ and $\mathcal{I}$ be the set of variables and instructions in the program respectively. Also, let $\mathcal{T}$ and $V$ be the set of terms and program points; and $\mathcal{D}(V, \mathcal{F})$ be the corresponding data flow framework. Let $\mathcal{W} \subseteq \mathcal{T}$ be the corresponding set of terms of length atmost two (in terms of number of operands).

The basic motive behind this approach was to prevent the disappearance of some equivalence classes in between the program due to reassignment to variables (see \autoref{sec:tc4} for an example). The classes disappear because there exists no expression in our working set $\mathcal{W}$ that belongs to that class however that class still contains expressions outside $\mathcal{W}$.

\subsection{Augmented Program}
\label{subsec:AugmentedProgram}

Every instruction in the original program is extended to a \textbf{bundle} of instructions - for each instruction $i \in \mathcal{I}$, add immediately after it dummy instructions of the form $x_i \gets x, \forall x \in \mathcal{X}$. This new program with dummy instructions is called the \textbf{augmented program} and the new variables introduced are called \textbf{shadow variables}.
\\
Let $\overline{\mathcal{X}}$, $\overline{\mathcal{I}}$, $\overline{\mathcal{T}}$ and $\overline{V}$ be the corresponding sets of variables, instructions, terms and program points respectively; and $\overline{\mathcal{D}}(\overline{V}, \overline{\mathcal{F}})$ be the corresponding data flow framework. Also $\overline{\mathcal{W}} \subseteq \overline{\mathcal{T}}$ be the corresponding set of terms restricted to length atmost two.

\subsection{Idea}
\label{subsec:Idea}
Let $\mathcal{L}$ and $\overline{\mathcal{L}}$ be the infinite lattices associated with the original and augmented programs respectively. These are same as $\overline{\mathcal{G}(\mathcal{T})}$ described in \autoref{sec:CongruenceRelation}, for the original and augmented programs.
\\
Given a partition $\mathcal{P}$, define 
$$\Phi_W(\mathcal{P}) = \{A_i \cap W \mid A_i \in \mathcal{P} \ \mathrm{and} \ (A_i \cap W) \neq \phi\}$$
Let $\Phi_W$ be generalised so that when applied to set of partitions $\mathcal{S}$,
$$\Phi_W(\mathcal{S}) = \{\Phi_W(\mathcal{P}) \mid \mathcal{P} \in \mathcal{S}\}$$
It is clear that $\mathcal{L} = \Phi_{\mathcal{T}}(\overline{\mathcal{L}})$. Let $\mathcal{L}_{\mathcal{W}} = \Phi_{\mathcal{W}}(\mathcal{L})$ and $\overline{\mathcal{L}}_{\overline{\mathcal{W}}} = \Phi_{\overline{\mathcal{W}}}(\overline{\mathcal{L}})$ be the finite lattices associated with the original and augmented lattices respectively.
\\
Let $f_{\mathcal{D}}$ and $\overline{f}_{\overline{\mathcal{D}}}$ be be the composite transfer function associated $\mathcal{L}$ and $\overline{\mathcal{L}}$ respectively, as defined in \autoref{chap:chapter3}. The Hebrand congruence functions are given by 
$$\mathcal{H}_{\mathcal{D}} = \bigwedge \{\top, f_{\mathcal{D}}(\top), f_{\mathcal{D}}^2(\top), ...\}\ \mathrm{and}\ \overline{\mathcal{H}}_{\overline{\mathcal{D}}} = \bigwedge \{\overline{\top}, \overline{f}_{\overline{\mathcal{D}}}(\overline{\top}), \overline{f}_{\overline{\mathcal{D}}}^2(\overline{\top}), ...\}$$
for the original and augmented lattice. Here, $\top$ and $\overline{\top}$ stands for top element in the product lattices corresponding to $\mathcal{L}$ and $\overline{\mathcal{L}}$ respectively.

\subsubsection{Observations}
\label{subsubsec:Observations}
\begin{itemize}
    \item $\Phi_{\mathcal{W}}(\mathcal{H}_{\mathcal{D}}) = \Phi_{\mathcal{W}}(\overline{\mathcal{H}}_{\overline{\mathcal{D}}})$
    \item $\Phi_{\mathcal{W}}(\Phi_{\overline{\mathcal{W}}}(\overline{\mathcal{H}}_{\overline{\mathcal{D}}})) = \Phi_{\mathcal{W}}(\mathcal{H}_{\mathcal{D}})$
\end{itemize}

\subsubsection{Claim}
\label{subsubsec:Claim}
$$\Phi_{\overline{W}}(\overline{\mathcal{H}}_{\overline{\mathcal{D}}}) =\text{ Maximum fix point of transfer function } \overline{f}_{\overline{\mathcal{W}}} \text{ on } \overline{\mathcal{L}}_{\overline{\mathcal{W}}}$$
The right hand side can computed efficiently using the algorithms in \autoref{chap:chapter4}, on the augmented program.

\subsection{Hypothesis}
\label{subsec:Hypothesis}
Following hypothesis was proposed which was supposed to be used later to prove the claim - suppose at program point $v \in V$ in the original program, variable $x \cong t$ (where $\mid t \mid\ \geq 2$) in our original infinite lattice $\mathcal{L}$, then at the output of the bundle corresponding to program point $v$ in the augmented program, there exists $t'$ in augmented finite lattice $\overline{\mathcal{L}}_{\overline{\mathcal{W}}}$ with $\mid t' \mid\ = 2$ which is equivalent to $x$ and consists only of constants and shadow variables.
\\
However, a counter example to the hypothesis was discovered while attempting the proof for the hypothesis. Consider test case in \autoref{sec:tc13} - at the confluence point (program point 6), in the original infinite lattice $x$ and $((y + z) + n)$ are equivalent; but in the augmented finite lattice no two length expression consisting of only shadow variables and constants that is equivalent to $x$.

\section{Why the Algorithm Works}
\label{sec:WhyTheAlgorithmWorks}
Instead of a general case, consider the ones in which the program has no loops. In such cases the control flow graph (CFG) is a directed acyclic graph (DAG). And if the algorithm processes CFG nodes in the topological order, it converges in the first iteration itself.

The \texttt{Parent} map captures the syntactic information present in the expressions in the form of an ordered tree. Each node in the tree is assigned a unique set identifier and represents some syntactically equivalent expressions. Suppose a node has set identifier $x$ and \texttt{Partitions[v][e]} is $x$ - it means that at program point $v$, expression $e$ has same syntactic value (across all program paths from the start) as represented by the node. A node captures this syntactic information in the form of its two ordered childrens and an operator - which is precisely the information stored the \texttt{Parent} map. Leaf nodes represents atomic values - a constant, non-deterministic or variable (variable in terms of different values along different execution paths) value. These properties implies that two expressions that are assigned the same set identifier at a program point are Herbrand equivalent at that point, which shows the correctness of the algorithm.

Now induction can be used to show that the algorithm actually maintains the properties mentioned in the last paragraph when it processes the nodes in the topological order.
\\
At the \texttt{START} point all the expressions in $\mathcal{W}$ are inequivalent to each other. And the algorithm also assigns different set identifiers to elements of $\mathcal{W}$. Also, for all $x_1 + x_2 \in (\mathcal{W} \setminus (\mathcal{C} \cup \mathcal{X}))$, \texttt{Parent[\{+, Partitions[START][$x_1$], Partitions[START][$x_2$]\}]} is assigned \texttt{Partitions[START][$x_1 + x_2$]}. So the properties hold at \texttt{START} point.

Now the properties will be shown to hold for program point $v$, assuming that they hold at all points which occur before $v$ in the topological order - in particular they hold for all $u \in \texttt{Predecessors[v]}$.
\begin{itemize}
    \item \textbf{$v$ is a transfer point}\\
    Suppose $v$ represents the assignment $x = e$ and $u$ is its predecessor. With respect to $u$, the only expressions changing values are the ones which contain $x$. So \texttt{Partitions[u]} is copied to \texttt{Partitions[v]} and $x$ is assigned the set identifier of $e$ because they now have equal values.\\
    Now consider all expressions of the form $a + b$ which contains $x$. If \texttt{Parent[\{+, Partitions[v][a], Partitions[v][b]\}]} is defined it means that there already exists a node that represents expressions syntactically equivalent to the value of $a + b$ at the current point. So \texttt{Partitions[v][a + b]} should be assigned the same set identifier. However, if no such node exists a new one needs to be created and is to be assigned to \texttt{Parent[\{+, Partitions[v][a], Partitions[v][b]\}]} as well as to \texttt{Partitions[v][a + b]} to maintain the properties. This is consistent with what the algorithm actually does.

    \item \textbf{$v$ is a confluence point}\\
    If $e \in \mathcal{W}$ has same set identifiers for all $u \in \texttt{Predecessors[v]}$, it means that it has same syntactic value at all its predecssors and hence even at the confluence point - so it must be assigned the same set identifier (as that of the predecessors) even at $v$ to maintain the properties.\\
    However, if $e \in \mathcal{W}$ has different identifiers at the predecessors then not all the values at the predecessor points are syntactically equivalent - for this case we require a new leaf node (because of variable values at predecessors). So we must find expressions which are equivalent to $e$ at all the predecessors - and assign a new set identifier (new leaf node) for them to maintain the properties.\\
    These both cases are consistent with the algorithm.
\end{itemize}
