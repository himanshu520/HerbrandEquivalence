\chapter{Algorithm for Herbrand Analysis}
\label{chap:chapter4}

This chapter presents an algorithm for Herbrand analysis. The pseudocode  
mentioned here is an updated version of the algorithm mentioned in \cite{Babu} 
for Herbrand equivalence analysis. The corresponding implementation done for 
LLVM compiler framework and a toy language can be found on 
\href{https://www.github.com/himanshu520/HerbrandEquivalence}{GitHub}.
Also, see \autoref{chap:chapter7} and \autoref{chap:chapter8} for more details.

One important distinction must be clear between the general Herbrand analysis problem
and the one that the algorithm in this chapter addresses. The universe for the Herbrand 
equivalence problem is the set of all expressions that can be formed using constants, 
variables and operators used in the program. But as already mentioned, the algorithm 
here is concerned with a restricted universe - set of expressions of length atmost two 
formed using constants, variables and operators in the program. 

\section{Notation}
\label{sec:NotationPseudocode}

Let $\mathcal C$ and $\mathcal X$ be the set of constants and variables used in the program;
$\mathcal W$ be our working set, which is the set of all expressions of length at most two 
that can be formed using ($\mathcal C \cup \mathcal X$). Also, $V$ be the set of all program 
points, with a special point \texttt{START} denoting the beginning of the program.
\bigbreak \noindent Three global variables are maintained - 
\begin{itemize} \tightlist
    \item \texttt{Partitions} : It is two-dimensional integer array indexed by the 
                                elements of sets $V$ and $\mathcal W$ respectively. It 
                                helps to keep track of partitions at some $v \in V$ by
                                holding same integer at \texttt{Partitions[$v$][$e$]} and
                                \texttt{Partitions[$v$][$e'$]}, for $e, e' \in \mathcal W$ 
                                if they belongs to the same equivalence class at $v$.
                                Basically, the array maintains set identifiers which helps 
                                to identify whether two expressions belong to the same sets 
                                (equivalence classes).
    \item \texttt{SetCnt} : It helps to keep track of the next set identifier (an integer)
                            to be used. Whenever, a set identifier is needed the current 
                            value of \texttt{SetCnt} is used and at the same time it is 
                            incremented so that the same number is never used again.
    \item \texttt{Parent} : It is \texttt{map} indexed by a tuple of three elements - an   
                            operator and two set identifiers. Whenever an expression 
                            ($x + y$) is assigned a set identifier $c$, $c$ is stored in 
                            \texttt{Parent} with $\{+, a, b\}$ as key, where $a$ and $b$ are 
                            the set identifiers of $x$ and $y$ respectively. Next time 
                            when a set identifier for an expression ($x' + y'$) is needed, 
                            where $x'$ and $y'$ have identifiers $a$ and $b$ respectively, 
                            $c$ is used instead of using a new set identifier - the \texttt
                            {Parent} of $\{+, a, b\}$ that was earlier stored in the map.
\end{itemize}

\bigbreak \noindent There are a few functions whose definitions are not required explicitly - 
\begin{itemize} \tightlist
    \item \texttt{OPERATOR($e$)} : Returns operator used in expression $e \in (\mathcal W \setminus (\mathcal C \cup \mathcal X))$
    \item \texttt{LEFT($e$)} : Returns left operand of expression $e \in (\mathcal W \setminus (\mathcal C \cup \mathcal X))$
    \item \texttt{RIGHT($e$)} : Returns right operand of expression $e \in (\mathcal W \setminus (\mathcal C \cup \mathcal X))$
    \item \texttt{PREDECESSORS($v$)} : Returns the set of predecessors of program point $v \in V$
\end{itemize}

\bigbreak \noindent \textbf{NOTE} - For implementation, the identifiers corresponding to 
$\top$ partition should be such that they don't occur in normal partitions and are also 
easily distinguishable from them. An easy choice for consistency is to use non-negative 
integers as normal set identifiers and an array of -1 to refer $\top$ partition.

\section{Pseudocode}
\label{sec:Pseudocode}

\begin{algorithm}
    \caption{Main Herbrand Equivalence Analysis Function}
    \label{alg:HerbrandEquivalenceAnalysis}
    \begin{algorithmic}
        \Procedure{HerbrandAnalysis}{$ $}
            \State \Comment{\% Initialise \texttt{Partitions} for all program points \%}
            \For{$v \in V$}
                \State \texttt{Partitions[$v$]} $\gets \top$
            \EndFor
            \State \Comment{\% Update \texttt{Partitions} for \texttt{START} point \%}
            \State \Call{findInitialPartition}{$ $}
            \State \Comment{\% Process all program points till convergence \%}
            \State \textbf{converged} $\gets$ \texttt{false}
            \While{\texttt{converged} \textit{is False}}
                \State \texttt{converged $\gets$ true}
                \For{$v \in (V \setminus \{\texttt{START}\})$}
                    \State \textbf{oldPartition} $\gets$ \texttt{Partitions[$v$]}
                    \State \Comment{\% Update \texttt{Partitions} at $v$ \%}
                    \If{$v$ \textit{is a Transfer Point}}
                        \State \Call{TransferFunction}{$v$}
                    \Else
                        \State \Call{ConfluenceFunction}{$v$}
                    \EndIf
                    \State \Comment{\$ Update \texttt{convergence} flag \%}
                    \If{\textit{not} \Call{SamePartition}{\texttt{oldPartition, Partitions[$v$]}}}
                        \State \texttt{converged $\gets$ false}
                    \EndIf
                \EndFor
            \EndWhile
        \EndProcedure
    \end{algorithmic}
\end{algorithm}

\begin{algorithm}
    \caption{Transfer Function}
    \label{alg:TransferFunction}
    \begin{algorithmic}
        \Procedure{TransferFunction}{$v : x \gets e$}
            \State \textbf{u} $\gets$ \Call{Predecessors}{$v$}
            \State \texttt{Partitions[$v$] $\gets$ Partitions[$u$]}
            \State \Comment{\% Update set identifier for $x$ \%}
            \If{$e$ \textit{is Deterministic}}
                \State \texttt{Partitions[$v$][$x$] $\gets$ Partitions[$v$][$e$]}
            \Else 
                \State \texttt{Partitions[$v$][$x$] $\gets$ SetCtr++}
            \EndIf
            \State \Comment{\% Update set identifiers for expressions containing $x$ \%}
            \For{$\{e' \in (\mathcal W \setminus (\mathcal C \cup \mathcal X)) \mid x \in e'\}$}
                \State \texttt{Partitions[v][$e'$]} $\gets$ \Call{GetSetId}{$v$, $e'$}
            \EndFor
        \EndProcedure
    \end{algorithmic}
\end{algorithm}

\begin{algorithm}
    \caption{Confluence Function}
    \label{alg:ConfluenceFunction}
    \begin{algorithmic}
        \Procedure{ConfluenceFunction}{$v$}
            \State \Comment{If all predecessor partitions are $\top$ then current partition will also be $\top$}
            \State \textbf{continueFlag} $\gets$ false
            \For{$u \in$ \Call{Predecessors}{$v$}}
                \If{\texttt{Partitions[$u$] $\neq \top$}}
                    \State \texttt{continueFlag $\gets$ true}
                \EndIf
            \EndFor
            \State
            \If{\texttt{continueFlag} \textit{is False}}
                \State \texttt{Partitions[$v$] $\gets \top$}
                \State \Return{}
            \EndIf
            \State \Comment{\texttt{accessFlag} keeps track of processed expressions}
            \For{$e \in \mathcal W$}
                \State \textbf{accessFlag}[$e$] $\gets$ \texttt{false}
            \EndFor
            \State \Comment{Process all expressions if they are still unprocessed}
            \For{$e \in \mathcal W$}
                \If{\texttt{accessFlag[$e$]} \textit{is False}}
                    \State \Comment{\texttt{PredIDs} is the set of set identifiers of $e$ at its predecessors}
                    \State \textbf{PredIDs} $\gets \phi$
                    \For{$u \in$ \Call{Predecessors}{$v$}}
                        \If{\texttt{Partitions[u] $\neq \top$}}
                            \State \texttt{PredIDs $\gets$ (PredIDs $\cup$ Partitions[$u$][$e$])}
                        \EndIf
                    \EndFor
                    \State
                    \If{\texttt{PredIDs} \textit{is Singleton}}
                        \State \texttt{Partitions[$v$][$e$] $\gets$ PredIDs}
                        \State \texttt{accessFlag[$e$] $\gets$ true}
                    \Else
                        \State \Comment{\texttt{expClass} holds $e' \in \mathcal W$ that are equivalent to $e$ at all predecessors of $v$}
                        \State \textbf{expClass} $\gets \mathcal W$
                        \For{$u \in$ \Call{Predecessors}{$v$}}
                            \If{\texttt{Partitions[u] $\neq \top$}}
                                \State \texttt{expClass $\gets$ (expClass} $\cap$ \Call{GetClass}{$u$, $e$})
                            \EndIf
                        \EndFor
                        \State \Comment{Update \texttt{Partitions} map}
                        \State \textbf{newSetID} $\gets$ \texttt{SetCnt++}
                        \For{\texttt{$e' \in$ expClass}}
                            \State \texttt{Partitions[$v$][$e'$] $\gets$ newSetID}
                            \State \texttt{acessFlag[$e'$] $\gets$ true}
                        \EndFor
                    \EndIf
                \EndIf
            \EndFor
            \State
            \State \Comment{Update \texttt{Parent} map}
            \For{$e \in (\mathcal W \setminus (\mathcal C \cup \mathcal X))$}
                \State \textbf{op} $\gets$ \Call{Operator}{$e$}
                \State \textbf{leftSetID} $\gets$ \texttt{Partitions}[\Call{Left}{$e$}]
                \State \textbf{rightSetID} $\gets$ \texttt{Partitions}[\Call{Right}{$e$}]
                \State \texttt{Parent[\{op, leftSetID, rightSetID\}] $\gets$ Partitions[v][$e$]}
            \EndFor
        \EndProcedure
    \end{algorithmic}
\end{algorithm}

\begin{algorithm}
    \caption{Checks whether two partitions are same or not}
    \label{alg:SamePartition}
    \begin{algorithmic}
        \Procedure{SamePartition}{\texttt{first, second}}
            \For{$e \in \mathcal W$}
                \If{\Call{GetClass}{\texttt{first}, $e$} $\neq$ \Call{GetClass}{\texttt{second}, $e$}}
                    \State \Return{\texttt{false}}
                \EndIf
            \EndFor
            \State \Return{\texttt{true}}
        \EndProcedure
    \end{algorithmic}
\end{algorithm}

\begin{algorithm}
    \caption{Finds equivalence class of an expression in a partition}
    \label{alg:GetClass}
    \begin{algorithmic}
        \Procedure{GetClass}{\texttt{partition}, $e$}
            \State \textbf{expClass} $\gets \phi$
            \For{$e' \in \mathcal W$}
                \If{\texttt{partition[$e'$] == partition[$e$]}}
                    \State \texttt{expClass $\gets$ (expClass $\cup$ $\{e'\}$)}
                \EndIf
            \EndFor
            \State \Return{\texttt{expClass}}
        \EndProcedure
    \end{algorithmic}
\end{algorithm}

\begin{algorithm}
    \caption{Initialises partition for START point}
    \label{alg:FindInitialPartition}
    \begin{algorithmic}
        \Procedure{FindInitialPartition}{$ $}
            \For{$x \in (\mathcal C \cup \mathcal X)$}
                \State \texttt{Partitions[START][$x$] $\gets$ SetCtr++}
            \EndFor
            \For{$e \in (\mathcal W \setminus (\mathcal C \cup \mathcal X))$}
                \State \texttt{Partitions[START][$e$]} $\gets$ \Call{GetSetID}{\texttt{Partitions[START], $e$}}
            \EndFor
        \EndProcedure
    \end{algorithmic}
\end{algorithm}

\begin{algorithm}
    \caption{Finds and returns set identifier for a two length expression in a partition, by
             looking at its operands and operator}
    \label{alg:GetSetID}
    \begin{algorithmic}
        \Procedure{GetSetID}{\texttt{partition, $e$}}
            \State \textbf{op} $\gets$ \Call{Operator}{$e$}
            \State \textbf{leftSetID} $\gets$ \texttt{partition}[\Call{Left}{$e$}]
            \State \textbf{rightSetID} $\gets$ \texttt{partition}[\Call{Right}{$e$}]
            \State
            \If{\textit{not defined} \texttt{Parent[\{op, leftSetID, rightSetID\}]}}
                \State \texttt{Parent[\{op, leftSetID, rightSetID\}] $\gets$ SetCtr++}
            \EndIf 
            \State
            \State \Return{\texttt{Parent[\{op, leftSetID, rightSetID\}]}}
        \EndProcedure
    \end{algorithmic}
\end{algorithm}

\section{Updates Over Original Algorithm}
\label{sec:UpdatesOverOriginalAlgorithm}
Following major improvements/corrections have been made to the original algorithm 
mentioned in \cite{Babu}.
\begin{itemize} \tightlist
    \item \textbf{Representing Partitions}\\
    The original algorithm uses \texttt{ID structure} to maintain equivalence information. A two-dimensional array \texttt{Partitions} having an entry for each program point and each expression, stores a pointer to an \texttt{ID object}. At a program point, two expressions are equivalent if they contain pointers to the same object. Though this approach is correct, it adds a lot of overhead in the implementation both in terms of time and space - the \texttt{ID object}s have to be created and destroyed during runtime, their data fields have to be maintained properly etc.\\
    The updated algorithm solves this problem by using just integer set identifiers instead of pointers to some dynamically created objects and the \texttt{Parent map}. Again same as before, two expressions are equivalent at a program point if \texttt{Partitions} array stores same set identifiers for the expressions at that program point. \texttt{Parent} map captures the relation between the set identifiers. This modification makes the actual implementation to be more simpler, efficient and intuitive.
    \item \textbf{Confluence Function}\\
    The \texttt{Confluence} function in the original algorithm processes only the set $\mathcal C \cup \mathcal V$ in its main loop. This is wrong and instead the whole working set $\mathcal W$ should be considered.\\
    As an example, consider the test case in \autoref{sec:tc15}. At the confluence point, $y$ and $x + 2$ must be equivalent, but the original algorithm assigns them pointers to different \texttt{ID objects} indicating they are not equivalent.
    \item \textbf{Non-deterministic Assignment}\\
    Non-deterministic assignment need not be handled separately and a single transfer function is sufficent (see \autoref{alg:TransferFunction}).
\end{itemize}
