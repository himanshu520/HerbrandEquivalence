\chapter{Toy Language Implementation of the Algorithm}
\label{chap:chapter8}

This chapter provides details of \texttt{C++} implementation of the Herbrand 
equivalence algorithm for a custom toy language. The actual implementation along 
with other details like how to run and how to interpret the output etc. can be 
found on \href{https://github.com/himanshu520/HerbrandEquivalence/tree/master/ToyLanguage}{GitHub} 
and for extensive documentation refer to \href{https://himanshu520.github.io/HerbrandEquivalenceToyDocs/}{ToyLanguageDocs}.

\section{The Toy Language}
\label{sec:TheToyLanguage}

There are three different kinds of statements in a toy language program.

\begin{itemize} \tightlist
    \item \textbf{Normal Instructions}\\
    These are of one of the following three forms where \texttt{x} should be a variable, \texttt{y} and \texttt{z} can be either an integer or a variable. The last one specifies \textbf{non-deterministic assignment}.
    \begin{itemize} \tightlist
        \item \texttt{x = y}
        \item \texttt{x = y + z}
        \item \texttt{x = *}
    \end{itemize}
    Variables do not require declaration for use.

    \item \textbf{\texttt{LABEL} Statements}
    \begin{itemize} \tightlist
        \item Such statements define labels which act as alias for their 
        immediate next statement of the \textbf{first kind} (ie. the normal 
        instructions). A \texttt{LABEL} statement occurring at the end refers 
        to the \textbf{end of the program}.
        \item It has following format - \texttt{'LABEL identifier1 identifier2 ...'}.
        \item A normal instruction can either have a single label or multiple labels. 
        Multiple labels can either be defined in the same \texttt{LABEL}
        statement or different statements.
        \item All the \texttt{LABEL} identifiers used in the definitions must 
        be unique (even those used for the same instruction).
    \end{itemize}

    \item \textbf{\texttt{GOTO} Statements}
    \begin{itemize} \tightlist
        \item \texttt{GOTO} statements are used to specify jumps - the 
        successors of immediately preceding instruction of the \textbf{first 
        kind} (ie. the normal instructions). A \texttt{GOTO} statement at the 
        beginning means two different control flow paths from the \textbf{start 
        of the program}.
        \item It has following format - \texttt{'GOTO identifier1 identifier2 ...'}.
        \item Again like \texttt{LABEL} statements, for specifying multiple 
        successors of a normal instruction there can either be a single \texttt{GOTO} 
        statement with multiple label identifiers or multiple \texttt{GOTO} 
        statements.
        \item For Herbrand equivalence analysis the condition of the jumps are 
        irrelavent, so in the toy language all the jumps are unconditional.
        \item For instructions of the \textbf{first kind} without \texttt{GOTO},
        by default their successor is the immediate occurring instruction of 
        the \textbf{first kind}. However, if such instructions has a 
        corresponding \texttt{GOTO}, all the successors \textbf{must} be 
        specified there itself - they do not have any default successors.
        \item The labels used in the \texttt{GOTO} statement must be defined in 
        some \texttt{LABEL} statement in the program.
    \end{itemize}
\end{itemize}

\paragraph{Other important points}
\begin{itemize} \tightlist
    \item Each instruction must be specified a single line and should be the
    only instruction in that line.
    \item Empty lines in the program text are ignored.
    \item A normal instruction can contain arithmatic operators other than \texttt{+}.
    \item A variable name must start with an alphabetic character.
    \item Specify the program properly with atleast one whitespace between the
    tokens. Also check whether the program is parsed properly by looking at
    the output when the program is executed.
    \item Variables and constants defined in any unreachable instruction are
    also considered for analysis, but the instruction itself is ignored.
\end{itemize}

\section{\texttt{MapVector.h}}
\label{sec:MapVectorH}
This file defines a \texttt{MapVector} data structure for bi-directional 
mapping to integers. Internally it uses \texttt{std::map} for forward mapping 
to integers and \texttt{std::vector} for reverse mapping. The data structure is 
generalised using \texttt{templates}.

\subsection{Private Data Members}
\label{subsec:PrivateDataMembersMapVectorH}
\begin{itemize} \tightlist
    \item \texttt{Map} - Internal \texttt{std::map} variable for forward mapping.
    \item \texttt{Vector} - Internal \texttt{std::vector} variable for reverse mapping.
\end{itemize}

\subsection{Public Member Functions}
\label{subsec:PublicMemberFunctionsMapVectorH}
\begin{itemize} \tightlist
    \item \texttt{MapVector()} - Default constructor.
    \item \texttt{begin()} - Returns forward iterator to the beginning of \texttt{Vector}.
    \item \texttt{end()} - Returns forward iterator to the end of \texttt{Vector}.
    \item \texttt{rbegin()} - Returns reverse iterator to the end of \texttt{Vector}.
    \item \texttt{rend()} - Returns reverse iterator to the beginning of \texttt{Vector}.
    \item \texttt{empty()} - Returns boolean to indicate whether the \texttt{MapVector} object is empty.
    \item \texttt{size()} - Returns the number of unique items inserted into \texttt{MapVector} object.
    \item \texttt{insert(item)} - Inserts \texttt{item} in the \texttt{MapVector} object if not already present.
    \item \texttt{getInt(item)} - Returns the integer to which \texttt{item} is forward mapped.
    \item \texttt{operator[](n)} - Returns \texttt{item} which is forward mapped to integer $n$.
    \item \texttt{clear()} - Clears the \texttt{MapVector} object by deleting existing items.
\end{itemize}

\subsection{How it works}
\label{subsec:HowItWorks}
When an \texttt{item} is to be inserted in a \texttt{MapVector} object, first 
a check is made if the \texttt{item} is already present. If not found, 
it is inserted into the internal \texttt{Map} data member, being forward mapped 
to next available integer $x$ starting from $0$ (which is same as \texttt{Vector.
size()} or \texttt{Map.size()} before insertion). Also at the same time, it is 
inserted into the \texttt{Vector} data member at the same integer index $x$ for 
reverse mapping.\\
So, \texttt{MapVector} provides $\mathcal{O}(\log{}n)$ forward mapping and 
$\mathcal{O}(1)$ reverse mapping, where $n$ is the size of the 
\texttt{MapVector} object. Insertion operation is also $\mathcal{O}(\log{}n)$.

\section{\texttt{Program.h}}
\label{ProgramH}
This file defines a \texttt{Program} data structure for capturing a toy 
language program. In addition it defines a \texttt{CustomOStream} class whose 
object acts as a placeholder inside \texttt{Program} for providing 
\texttt{std::cout} like functionality for \texttt{Program} class.\\
The details of \texttt{CustomOStream} class are irrelavent and hence not 
mentioned. The following subsections provide details of \texttt{Program} class.

\subsection{Public Data Structures}
\label{subsec:PublicDataStructuresProgramH}
\begin{itemize} \tightlist
    \item \texttt{ValueTy} - Represents constants and variables in the program. It has following data members -
        \begin{itemize} \tightlist
            \item \texttt{isConst} - Boolean to indicate whether the 
            corresponding object represents a constant or a variable.
            \item \texttt{index} - Corresponding integer index for the constant 
            or variable (see \texttt{Constants} and \texttt{Variables} data 
            members of \texttt{Program} class).
        \end{itemize}
    
    \item \texttt{ExpressionTy} - Represents an expression. It has following data members -
        \begin{itemize} \tightlist
            \item \texttt{op} - Character to represent the operand in the expression.
            \item \texttt{leftOp} - \texttt{ValueTy} object to represent the left operand.
            \item \texttt{rightOp} - \texttt{ValueTy} object to represent the right operand.
        \end{itemize}

    \item \texttt{InstructionTy} - Represents a normal instruction in the program. It has following data members - 
        \begin{itemize} \tightlist
            \item \texttt{lValue} - \texttt{ValueTy} object for lvalue of the instruction.
            \item \texttt{rValue} - \texttt{ExpressionTy} object for rvalue of the instruction.
            \item \texttt{reachable} - Boolean to indicate whether the instruction is reachable.
            \item \texttt{cfgIndex} - Index of the node in control flow graph for which the instruction defines the transfer function (see \texttt{CFG} data member of \texttt{Program} class).
            \item \texttt{predecessors} - \texttt{std::set<int>} of indexes of
            predecessors of the instruction (see \texttt{Instructions} data member of \texttt{Program} class).
        \end{itemize}
    
    \item \texttt{CfgNodeTy} - Represents a control flow graph node corresponding to the program. It has following data members -
        \begin{itemize} \tightlist
            \item \texttt{predecessors} - \texttt{std::vector<int>} object of indexes of predecessor control flow graph nodes (see \texttt{CFG} data member of \texttt{Program} class).
            \item \texttt{instructionIndex} - Index of the instruction that defines the transfer function if the node represents a transfer point (see \texttt{Instructions} data member of \texttt{Program} class).
        \end{itemize}
\end{itemize}

\subsection{Public Data Members}
\label{subsec:PublicDataMembersProgramH}
\begin{itemize} \tightlist
    \item \texttt{Variables} - \texttt{MapVector<string>} to store program variables.
    \item \texttt{Constants} - \texttt{MapVector<int>} to store program constants.
    \item \texttt{Instructions} - \texttt{std::vector<InstructionTy>} object to store program instructions.
    \item \texttt{CFG} - \texttt{std::vector<CfgNodeTy>} to store program control flow graph.
    \item \texttt{cout} - \texttt{CustomOStream} object to provide \texttt{std::cout} like functionality for \texttt{Program} class.
\end{itemize}

\subsection{Public Member Functions}
\label{subsec:PublicMemberFunctionsProgramH}
\begin{itemize}
    \item \texttt{parse(filename)} - Parses toy language program in 
    \texttt{filename} and updates \texttt{Variables}, \texttt{Constants} and 
    \texttt{Instructions} data members.\\
    It performs two passes over the program. In the first pass, it stores the 
    instructions, performs basic checks to ensure syntactic correctness, 
    determines labels and jumps. In the second pass it performs BFS from the 
    first instruction, mapping jumps to actual instructions and also 
    determining the reachability of instructions. The function \texttt{throw}s 
    an error in case any of the checks fails.
    \item \texttt{print()} - Prints the parsed program in readable format.
    \item \texttt{createCFG()} - Creates control flow graph corresponding to the
    parsed program by updating \texttt{CFG} data members. The control flow 
    graphs contains nodes corresponding to transfer points, confluence points, 
    \texttt{START} and \texttt{END}. Unreachable instructions are ignored while 
    creating the control flow graph.\\
    It also performs two passes over the program. In the first pass it assigns 
    indexes to the transfer points (reachable instructions) and confluence 
    points in \texttt{CFG} data member. And then in the second pass, it actually
    fills the \texttt{CFG} vector.
    \item \texttt{printCFG()} - Prints the control flow graph in readable format.
\end{itemize}

\paragraph{NOTES}
\begin{itemize} \tightlist
    \item \texttt{Program.h} also overloads \texttt{operator<} for 
    \texttt{ValueTy} and \texttt{ExpressionTy} so that \texttt{std::set} and 
    \texttt{std::map} objects can be created using them.
    \item For \texttt{std::cout} like facilities, \texttt{operator<<} is also 
    overloaded for \texttt{CustomOStream} to output \texttt{char}, 
    \texttt{int}, \texttt{string}, \texttt{ValueTy}, \texttt{ExpressionTy}, 
    \texttt{InstructionTy} objects.
\end{itemize}

\section{\texttt{HerbrandEquivalence.cpp}}
\label{sec:HerbrandEquivalenceCpp}
This file implements the Hebrand equivalence algorithm for the toy language  
described above.

\subsection{Global Variables}
\label{subsec:GlobalVariablesHerbrandEquivaleceCpp}
\begin{itemize} \tightlist
    \item \texttt{program} - \texttt{Program} object to store parsed toy program.
    \item \texttt{Ops} - \texttt{std::vector<char>} object to store the operators that can occur in the toy program being analysed.
    \item \texttt{IndexExp} - \texttt{std::map<Program::ExpressionTy, int>} to map program expressions of length atmost two to integers for indexing purpose.
    \item \texttt{SetCnt} - Next set identifier to be used.
    \item \texttt{Partitions} - \texttt{std::vector<std::vector<int>>} to keep track of partition at each program point. \texttt{Partitions[v][n]} stores set identifier for expression \texttt{e} such that \texttt{IndexExp[e]} is \texttt{n}, at program point represented by \texttt{program.CFG[v]}. \texttt{Partitions[v]} is the \textbf{partition vector} representing the partition at program point \texttt{program.CFG[v]}.
    \item \texttt{Parent} - \texttt{std::map<std::tuple<char, int, int>, int>} to store parent set identifiers.
\end{itemize}
\textbf{NOTE} - For more details on \texttt{SetCnt}, \texttt{Partitions} and \texttt{Parent} refer to \autoref{sec:NotationPseudocode}.

\subsection{Functions}
\label{subsec:FunctionsHerbrandEquivalenceCpp}
\begin{itemize} \tightlist
    \item \texttt{assignIndex()} - Updates \texttt{IndexExp} assigning integer indexes to expressions arbitrarily at the beginning of the analysis, for indexing purpose.
    \item \texttt{samePartition(p1, p2)} - Checks if the two partition vectors (\texttt{std::vector<int>}) \texttt{p1} and \texttt{p2} are same. They are same when values (ie. set identifiers) at two different indexes are equal in the first iff they are so in the second.
    \item \texttt{findSet(p, e)} - Finds set identifier representing expression \texttt{e} (which is of type \texttt{Program::ExpressionTy}) in partition vector \texttt{p}.
    \item \texttt{findInitialPartition(p)} - Intialises partition vector \texttt{p} with $\bot$.
    \item \texttt{getClass(p, n, c)} - Updates \texttt{c} (\texttt{std::set<int>}) to contain indexes of expressions which are equivalent to expression with index \texttt{n}, in the partition vector \texttt{p}.
    \item \texttt{printPartition(p)} - Prints partition vector \texttt{p} in readable format.
    \item \texttt{transferFunction(n)} - Applies corresponding transfer function to the partition at program point \texttt{program.CFG[n]}.
    \item \texttt{confluenceFunction(n)} - Applies the confluence operation to the partition at program point \texttt{program.CFG[n]}.
    \item \texttt{HerbrandEquivalence()} - Main Herbrand analysis function.
    \item \texttt{main()} - Parses toy program (\texttt{program.parse(argv[1])}), creates control flow graph (\texttt{program.createCFG()}) and then calls \texttt{HerbrandEquivalence()} function.
\end{itemize}

\bigskip \noindent \textbf{NOTE} - For more details on the function implementation, refer to \autoref{chap:chapter4} and for extensive documentation refer to \href{https://himanshu520.github.io/HerbrandEquivalenceToyDocs/}{ToyLanguageDocs}.
