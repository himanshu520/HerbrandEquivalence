\chapter{Introduction}
\label{chap:chapter1}
\pagenumbering{arabic}\hspace{3mm}

The basic job of a compiler is code translation from a high level 
language to a target assembly language. But, compilers also perform
multiple optimizations in the intermediate stages of translation, 
so that the finally generated code performs better than just a normal 
translated code. There might be a one time overhead of 
running optimizations, but the performance gain visible over multiple 
executions of the code outweighs it.

A typical modern day compiler performs a large number of optimizations 
like induction variable analysis, loop interchange, loop invariant code
motion, loop unrolling, global value numbering, dead code 
optimizations, constant folding and propagation, common subexpression 
elimination etc. One common feature of most of these optimizations is 
detecting equivalent program subexpressions. 

Checking semantic equivalence of program subexpressions has been shown 
to be an undecidable problem, even when all the conditional statements 
are considered as non deterministic. So in most of the cases, compilers 
try to find some restricted form of expression equivalence. One such
form of expression equivalence is \textbf{Herbrand equivalence} (see 
\autoref{sec:HerbrandEquivalence}).
Detecting equivalence of program subexpressions can be used for 
variety of applications. Compilers can use these to perform several 
of the optimizations mentioned earlier like constant propagation, 
common subexpression elimination etc. Program verification tools can 
use these equivalences to discover loop invariants and to verify 
program assertions. This information is also important for discovering
equivalent computations in different programs, which can be used by
plagiarism detection tools and translation validation tools \cite
{Necula, Pnueli}, which compare a program with an optimized version
in order to check correctness of the optimizer.


\section{Herbrand Equivalence}
\label{sec:HerbrandEquivalence}
A formal definition of \textbf{Herbrand equivalence} is given in 
\cite{Ruthing}.
Informally, two expressions are \textbf{Herbrand equivalent at a program 
point}, if and only if they have syntactically the same value at that 
given point, \textbf{across all the execution paths} from the start 
of the program which reaches that point. For the purpose of analysis, 
the operators themselves are treated as uninterprated functions with 
no semantic significance, only syntactic information is taken into 
consideration (see example in \autoref{sec:ASimpleExample}).

For \textbf{Herbrand equivalence analysis}, the universe is the set 
of all possible expressions that can be formed using the constants, 
variables and operators used in the program. And for each program 
point, partition it such that two expressions are Herbrand equivalent 
at that point if and only if they belong to the same partition class 
of that point.

\section{A Simple Example}
\label{sec:ASimpleExample}

\begin{figure}[!ht]
    \centering {
        \setlength{\fboxsep}{8pt}    
        \fbox{\includegraphics[scale=0.55]{HerbrandEquivalenceTrans.png}}
    }
    \caption{Example of Herbrand Equivalence}
    \label{fig:HerbrandEquivalenceTrans}
\end{figure}

\autoref{fig:HerbrandEquivalenceTrans} shows a simple example of Herbrand 
equivalence analysis. All the expressions that belongs to the same set at 
a program point are Herbrand equivalent at that point.

\begin{itemize}
    \item   Initially all the expressions are in separate sets, ie. 
    they are inequivalent to each other. In particular, note that 
    $X + 2$ and $2 + X$ are inequivalent because the operators are 
    being treated uninterprated with no semantic information of them, 
    which means there is no knowledge of commutativity of $+$.
    \item   After assignment $X = 2$, any occurrence of $X$ in an
    expression can equally be replaced with $2$. So, now all expressions 
    with $2$ in place of $X$ and vice versa are equivalent - that means
    $2 + 2$, $2 + X$, $X + 2$, $X + X$ are all equivalent - this still 
    is just syntactic information because $X$ and $2$ are equivalent. 
    However, if $4$ is also in the universe of expressions, $2 + 2$ and 
    $4$ are not equivalent as this is semantic information about operator $+$.
    \item   After assignment $Y = X$, any occurrence of $Y$ in 
    an expression can be replaced with $X$. Because $X$ and $2$ are already 
    equivalent, it means now $2$, $X$, and $Y$ are all equivalent to 
    each other. And two expressions are equivalent if one can be 
    obtained from the other by replacing one of these three with any of the 
    other two. For this example, it means that any two expressions of 
    the same length are equivalent.
\end{itemize}

\begin{figure}[!ht]
    \centering {
        \setlength{\fboxsep}{8pt}    
        \fbox{\includegraphics[scale=0.54]{HerbrandEquivalenceConv.png}}
    }
    \caption{Example of Herbrand Equivalence analysis at a confluence point}
    \label{fig:HerbrandEquivalenceConv}
\end{figure}

\autoref{fig:HerbrandEquivalenceConv} shows what happens at a 
\textbf{confluence point} - a point where multiple execution paths meet. 
Two expressions are Herbrand equivalent at the confluence point only if 
they are Herbrand equivalent at all the predecessor points.
\begin{itemize}
    \item   In the left branch $2$, $Y$, $Z$ are equivalent and 
    so are expressions which are interconvertible by replacement 
    of any of these three, with the other two.
    \item   The case with the right branch is similar, except $X$ is 
    equivalent to $2$ and $Z$ instead of $Y$.
    \item   At the confluence point, only $2$ and $Z$ are equivalent 
    because they are equivalent at both the predecessors points. $X$ 
    is equivalent to $2$ and $Z$ at the right predecessor but not 
    the left one and $Y$ is equivalent to $2$ and $Z$ at the left 
    predecessor but not the right. As before, expressions obtained by 
    replacing $2$ with $Z$ and vice versa are equivalent.
\end{itemize}

\section{Goal of the Project}
\label{sec:GoalOfTheProject}
Babu, Krishnan and Paleri \cite{Babu} has given an algorithm for 
Herbrand equivalence analysis restricted to program expressions. The 
basic goal of this project is to refine this general algorithm and 
then implement it for \href{https://llvm.org/}{Clang/LLVM compiler}. 
The implementation is also to be benchmarked using \href{https://www.spec.org/}
{SPEC} CPU benchmarks. Finally, a proof of correctness of the 
algorithm based on the theoretical work of Babu et al. \cite{Babu} 
has to be presented.

\section{Organization of the Report}
\label{sec:OrganizationOfTheReport}
\autoref{chap:chapter2} gives an overview of the previous works 
related to Herbrand equivalence; then summary of works of Babu et al. 
\cite{Babu} is specifically presented in \autoref{chap:chapter3} as 
it is closely related to the project. The updated algorithm for 
Herbrand equivalence analysis is given in \autoref{chap:chapter4}. 
\autoref{chap:chapter5} presents some important test cases on which 
the Herbrand analysis algorithm can be checked for correctness. 
\autoref{chap:chapter6} discusses about Clang/LLVM compiler and also 
provides a tutorial for writing LLVM optimzation passes. 
\autoref{chap:chapter7} and \autoref{chap:chapter8} explains the 
implementation of the algorithm done for Clang/LLVM compiler and a 
toy language respectively. Finally, \autoref{chap:chapter9} provides 
the details of the attempt made for proving the correctness of the 
algorithm. \autoref{chap:chapter10} concludes the report.
