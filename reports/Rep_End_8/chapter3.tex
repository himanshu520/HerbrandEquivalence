\chapter{Formulation by Babu, Krishnan and Paleri}
\label{chap:chapter3}

One of the problems with other former approaches to Herbrand 
equivalence is that most of the alogrithms are based on fix point 
computations. But the classical definition of Herbrand equivalence is 
not a fix point based definition making it difficult to prove their 
precision or completeness. Babu, Krishnan and Paleri \cite{Babu} gave 
a new lattice theoretic formulation of Herbrand equivalences and 
proved its equivalence to the classical version.

The paper defines a congruence relation on the set of all possible 
expressions and shows that the set of all congruences form a complete 
lattice. Then for a given dataflow framework with $n$ program points, 
a continuous composite transfer function is defined over the $n$-fold 
product of the above lattice such that the maximum fix point of the 
function yields the set of Herbrand equivalence classes at various 
program points. Finally, equivalence of this approach to the 
classical meet over all path definition of Herbrand equivalence is 
established.

Below is a brief summary of the developments in the paper, for more 
detailed approach and proofs and for equivalence to MOP characterization 
refer to \cite{Babu}.

\section{Program Expressions}
\label{sec:ProgramExpressions}

Let $\mathcal C$ and $\mathcal X$ be the set of constants and variables 
occurring in the program respectively. The program expressions (terms) 
can be described as 
$$t\ ::=\ c\ \mid\ x\ \mid\ t_1 + t_2$$
where $c \in \mathcal C$ and $x \in \mathcal X$.

\section{Congruence Relation}
\label{sec:CongruenceRelation}
Let $\mathcal T$ be the set of all program terms. A partition $\mathcal P$ of terms 
in $\mathcal T$ is said to be a congruence (of terms) if 
\begin{itemize} \tightlist
    \item For $t$, $t'$, $s$, $s'$ $\in$ $\mathcal T$, $t' \cong t$ and $s' \cong s$ iff $t' + s' \cong t + s$. 
    \item For $c \in \mathcal C$, $t \in \mathcal T$, if $t \cong c$ then either $t = c$ or $t \in \mathcal X$.
\end{itemize}
Let $\mathcal G(\mathcal T)$ be the set of all congruences over $\mathcal T$. 
An ordering is defined over $\mathcal G(\mathcal T)$ as $\mathcal P_1 \preceq \mathcal P_2$ for $\mathcal P_1, \mathcal P_2 \in \mathcal G(\mathcal T)$, if 
$\forall \mathcal A_1 \in \mathcal P_1, \ \exists \mathcal A_2 \in \mathcal P_2$ 
such that $\mathcal A_1 \subseteq \mathcal A_2$. \\
Also binary \textbf{confluence operation ($\land$)} is defined on $\mathcal G(\mathcal T)$ as
$$\mathcal P_1 \land \mathcal P_2\ =\ \{\mathcal A_i \cap \mathcal B_j\ \mid\ \mathcal A_i \in \mathcal P_1,\ \mathcal B_j \in \mathcal P_2 \text{ and } \mathcal A_i \cap \mathcal B_j \neq \phi\}$$
Finally $\mathcal G(\mathcal T)$ is extended to $\overline{\mathcal G(\mathcal T)}$ 
by introducing an abstract congruence $\top$ satisfying 
$\mathcal P \land \top = \top, \forall \mathcal P \in \overline{\mathcal G(\mathcal T)}$.
Also the congruence in which every element is in a separate class, is denoted as 
$\bot$.\\ 
With these definitions, $(\overline{\mathcal G(\mathcal T)}, \preceq, \bot, \top)$ 
forms a complete lattice, with $\land$ as its \textbf{meet operator}.

\section{Transfer Function}
\label{sec:TransferFunction}
An assignment $y := \beta$ transforms a congruence $\mathcal P$ to another 
congruence $\mathcal P'$. This can be described in the form of \textbf{transfer function} 
$f_{y := \beta}:\mathcal G(\mathcal T) \to \mathcal G(\mathcal T)$, given by
\begin{itemize} \tightlist
    \item $\mathcal B_i = \{t \in \mathcal T\ |\ t[y \leftarrow \beta] \in \mathcal A_i\}, \text{ for each } \mathcal A_i \in \mathcal P$
    \item $f_{y := \beta}(\mathcal P) = \{\mathcal B_i\ | \ \mathcal B_i \neq \phi\}$
\end{itemize}
This definition is extended to form \textbf{extended transfer function}, 
$\overline{f}_{y := \beta} : \overline{\mathcal G(\mathcal T)} \to \overline{\mathcal G(\mathcal T)}$ 
by defining $\overline{f}_{y := \beta}(\top)\ =\ \top$, otherwise 
$\overline{f}_{y := \beta}(\mathcal P) = f_{y := \beta}(\mathcal P)$.
The extended transfer function is \textbf{distributive}, \textbf{monotonic} and 
\textbf{continuous}.

\section{Non Deterministic Assignment}
\label{sec:NonDeterministicAssignment}
An assignment $y := *$ also transforms a congruence $\mathcal P$ to another 
congruence $\mathcal P'$. This can be described in the form of \textbf{transfer function} 
$f_{y := *}:\mathcal G(\mathcal T) \to \mathcal G(\mathcal T)$, given by
$\forall t, t' \in \mathcal T,\ t \cong_{f(\mathcal P)} t'$, (here 
$f(\mathcal P) = f_{y := *}(\mathcal P)$ for simplicity) iff
\begin{itemize} \tightlist
    \item $t \cong_{\mathcal P} t'$
    \item $\forall \beta \in (\mathcal T \setminus \mathcal T(y)),\ t[y \leftarrow \beta] \cong_{\mathcal P} t'[y \leftarrow \beta]$
\end{itemize}
As before this transfer function is extended to 
$\overline{f}_{y := *} : \overline{\mathcal G(\mathcal T)} \to \overline{\mathcal G(\mathcal T)}$ 
by defining $\overline{f}_{y := *}(\top)\ =\ \top$, otherwise 
$\overline{f}_{y := *}(\mathcal P) = f_{y := *}(\mathcal P)$. The function 
$\overline{f}_{y := *}$ is also \textbf{continuous}.

\section{Dataflow Analysis Framework}
\label{sec:DataflowAnalysisFramework}
A dataflow framework over $\mathcal T$ is $\mathcal D = (G, \mathcal F)$ where 
$G(V, E)$ is the control flow graph associated with the program and $\mathcal F$ 
is a collection of transfer function associated with program points.

\section{Herbrand Congruence Function}
\label{sec:HerbrandCongruenceFunction}
The Herbrand congruence function 
$\mathcal H_{\mathcal D} : V(G) \to \overline{\mathcal G(\mathcal T)}$ 
gives the Herbrand congruence associated with each program point and 
is defined to be \textbf{the maximum fix point} of the \textbf{continuous
composite transfer function} 
$f_{\mathcal D} : \overline{\mathcal G(\mathcal T)}^n \to \overline{\mathcal G(\mathcal T)}^n$, 
where $\overline{\mathcal G(\mathcal T)}^n$ is the product lattice, 
$f_{\mathcal D}$ is a function satisfying $\pi_k \circ f_{\mathcal D} = f_k$. Here $\pi_k$ is the projection map
and $f_k : \overline{\mathcal G(\mathcal T)}^n \to \overline{\mathcal G(\mathcal T)}$ 
is defined as follows 
\begin{itemize} \tightlist
    \item   If $k = 1$, the entry point of the program $f_k = \bot$.
    \item   If $k$ is a function point with \texttt{Pred}($k$) = $\{j\}$, then 
    $f_k = h_k \circ \pi_j$ where $h_k$ is the extended transfer function 
    corresponding to function point $k$.
    \item   If $k$ is a confluence point with \texttt{Pred}($k$) = $\{i, j\}$, 
    then $f_k = \pi_{i, j}$, where $\pi_{i, j}:\overline{\mathcal G(\mathcal T)}^n \to \overline{\mathcal G(\mathcal T)}$ 
    is given by $\pi_{i,j}(\mathcal P_1,\ \dots,\ \mathcal P_n) = \mathcal P_i \land \mathcal P_j$.
\end{itemize}
